\documentclass[12pt]{article}

\begin{document}

\title{The Brandiss Class and Package}
\author{Mario O.~Bourgoin}
\date{May 5, 2004}

\maketitle

\begin{abstract}
  The brandiss \LaTeX\ class and the associated package are intended
  to ease creating a dissertation in \LaTeX\ that meets the formatting
  requirements of the Brandeis Graduate School of Arts and Sciences,
  including those for on-line submission.
\end{abstract}

\section{Introduction}

The \emph{brandiss} \LaTeX\ class and the brandiss \LaTeX\ package are
intended to ease creating a dissertation in \LaTeX\ that meets the
formatting requirements of the Brandeis Graduate School of Arts and
Sciences (GSAS), including those for on-line submission.  The class
extends the \emph{amsbook} class of the AMS by setting the default
font size to 12 point, one-sided pagination, GSAS specified minimum
margins, center footer page numbering, and PDF compression level 0.
The package adds commands and environments for formatting the front
pages of the dissertation.

\section{Using The Brandiss Class and Package}

To use the class, copy the brandisssample.tex file and modify it to
include your dissertation.  If you want to have a hyperlinked document
and are experienced enough with \LaTeX\ to solve your own problems,
copy the brandisshyper.tex file and modify it to include your
dissertation.  If you want to create your own dissertation format
starting from the front pages defined for the brandiss class, you can
use the brandiss package.  In this case, the brandiss class definition
can help guide you towards the means of creating your format.

\section{Class and Package Options}

The brandiss class has all of the options of the amsbook class, and
two more: online vs.~printed and uncompressed vs.~compressed.  The
``online'' and ``printed'' options are passed to the brandiss package.
The ``uncompressed'' and ``compressed'' options are only meaningful if
a PDF file is being produced by ``pdflatex'', in which case the
uncompressed option means the resulting file is uncompressed and the
compressed option really means to not set the compression level to
none.  (That is, the compression level will be whatever is the
default.)

The brandiss package has only the online and printed options.  These
control the format of the signature page, where using the online
option formats it for on-line submission and the printed option
formats it for actual signatures.

\section{Required Classes and Packages}

The brandiss class requires the amsbook class and the brandiss, ifpdf,
setspace, geometry, and fancyhdr packages.  The brandiss package
receives the ``online'' and ``printed'' options, and the amsbook
package receives all remaining options.

The brandiss requires the setspace package.

\section{Page Format}

The brandiss class formats the page to meet GSAS margin and page
numbering requirements, and set the PDF compression level to 0 at the
beginning of the document if a PDF file is being produced.  The
geometry package is used to set the margins as follows:
\begin{center}
  \begin{tabular}{||c|c||}\hline\hline
    Margin & Length \\ \hline\hline
    Left & 1.5in \\ \hline
    Top & 1in \\ \hline\hline
  \end{tabular}
  \begin{tabular}{||c|c||}\hline\hline
    Margin & Length \\ \hline\hline
    Right & 1in \\ \hline
    Bottom & 1in \\ \hline\hline
  \end{tabular}
\end{center}
The brandiss class places the header and footer inside the margins and
increases the size of the header and its separation from the main text
to accommodate the header.  The fancyhdr package is used to define the
fancy page style to have the chapter number and name in the left of
the header on every page, and to have a page number centered in the
footer of every page.  The ifpdf package is used to detect when a PDF
is being produced, and an AtBeginDocument is issued to set the PDF
compress level to 0 if the uncompress option is active.

\section{Added Commands and Environments}

The brandiss class provides three commands in addition to those
provided by the required packages.  The commands and their effects
are:
\begin{itemize}
\item\textbf{$\backslash$frontmatter} Set double spacing, clear a double
  page, set the page style to plain, and the page numbering to roman.
\item\textbf{$\backslash$mainmatter} Clear a double page, set the page style
  to fancy, and the page numbering to arabic.
\item\textbf{$\backslash$backmatter} Clear a double page and set the page
  style to plain.
\end{itemize}

The brandiss package provide multiple commands and environments as
follows.  Many parameters must be set, and are used to include
information in the front pages of the dissertation.  The following
commands lets the user set those parameters.
\begin{itemize}
\item\textbf{$\backslash$disstitle} The title of the dissertation.
\item\textbf{$\backslash$dissauthor} The author.
\item\textbf{$\backslash$dissadvisor} The author's advisor.
\item\textbf{$\backslash$dissdepartment} The author's department.
\item\textbf{$\backslash$dissmonth} The graduation month.
\item\textbf{$\backslash$dissyear} The graduation year.
\item\textbf{$\backslash$dissdean} The dean of GSAS.
\end{itemize}
The following command provides transparent access to these values.
\begin{itemize}
\item\textbf{$\backslash$thediss} Access a parameter
  (i.e. ``$\backslash$thediss\{title\}'').
\end{itemize}
The following commands and environments format the front pages:
\begin{itemize}
\item\textbf{$\backslash$makedisstitle} This command formats the title
  page.
\item\textbf{$\backslash$disssignatures} This environment formats the
  signature page.  It is list-like and uses the
  \textbf{$\backslash$committeemember} command to separate committee
  member's names.  The advisor's name is automatically put as the
  first entry.  The format depends on the ``online'' and ``printed''
  options.
\item\textbf{$\backslash$disscopyright} This command formats the
  copyright page.
\item\textbf{$\backslash$dissdedication} This environment formats the
  dedication page.
\item\textbf{$\backslash$dissacknowledgments} This environment formats
  the acknowledgments page.
\item\textbf{$\backslash$dissabstract} This environment formats the
  abstract page.
\item\textbf{$\backslash$disspreface} This environment formats the
  preface page.
\end{itemize}

\end{document}

