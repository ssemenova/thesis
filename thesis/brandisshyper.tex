% This is a sample use of the brandiss class intended to ease meeting
% the GSAS formatting requirements for doctoral dissertations,
% including those for online deposit of dissertation.  Be sure to
% check that the output matches the current specifications.  The
% formatting requirements are on-line at:
%     http://www.brandeis.edu/gsas/students
% under ``Dissertation Guidelines.''
%
%         Take what you like,
%         Leave what you don't,
%         And put in something of your own.
%                  --Bruce Lee

% REQUIRED: Declare the document class to be used.  Note: To format
% the dissertation signature page for online submission, comment the first
% ``documentclass'' line and uncomment the following line.
\documentclass{brandiss}
%\documentclass[online]{brandiss}

% Optional: The graphicx package eases inserting pictures in documents.
\usepackage{graphicx}

% Optional: the amssymb package includes many useful mathematical symbols.
\usepackage{amssymb}

% REQUIRED: Hyperref should be the last package you load.  It
% generates hyperlinks for anything in the table of contents, figures,
% and bibliography.  For some reason, special commands must be used to
% make sure that the list of figures, list of tables, and biliography
% generate correct hyperlinks.
\usepackage{hyperref}
\hypersetup{% Generate hyperlinks for
  bookmarks,% the table of contents and
  hyperfigures,% the list of figures.
  plainpages=false,% REQUIRED: Number pages as formatted.
  pdfborder={0 0 0}% Optional: Links have no borders.
}

% Optional: Number sections and figures within chapters.
\numberwithin{section}{chapter}
\numberwithin{figure}{chapter}

% REQUIRED: Dissertation information.  This information can be
% accessed with the \thediss{} command, as is shown in the document
% below.
\disstitle{Dissertation Title}
\dissauthor{Your Name}
\dissadvisor{Student's Advisor}
\dissdepartment{Mathematics}
\dissmonth{May}% My graduation month.
\dissyear{2004}% My graduation year.
\dissdean{Adam Jaffe}% Dean of GSAS, May 2004.

% Information about your dissertation to be included in the properties of
% the PDF document.
\hypersetup{%
  pdftitle=\thediss{title},
  pdfauthor=\thediss{author},
  pdfsubject=Geometric Topology,
  pdfkeywords={Your keywords go here separated by commas.}
}

% Optional: Define environments to ease creating definitions, theorems
% and remarks.  For proofs, use the ``proof'' environment.
\theoremstyle{definition}
\newtheorem{definition}{Definition} 
\theoremstyle{plain}
\newtheorem{theorem}{Theorem} 
\theoremstyle{remark}
\newtheorem{remark}{Remark} 

% Optional: Commands to make LaTeX equations use mathematical terms.
\newcommand{\into}{\hookrightarrow}
\newcommand{\onto}{\twoheadrightarrow}

\newcommand{\rC}{\mathbb C} % The field of complex numbers.
\newcommand{\pCP}{\mathbb{CP}} % The complex projective space.
\newcommand{\rZ}{\mathbb Z} % The ring of integers.
\newcommand{\mM}{\mathcal M} % A manifold M.

\DeclareMathOperator{\Homol}{H} % The co/homology functor.
\newcommand{\Partial}[2]{\dfrac {\partial #1} {\partial #2}}
\newcommand{\Hessian}[3]{\dfrac {\partial^2 #1} {\partial #2\partial #3}}

% The actual document begins.
\begin{document}

% GSAS formatting requirements: Front page order is title, signature,
% copyright, dedication, acknowledgements, abstract, preface, table of
% contents, list of figures, list of tables.

% REQUIRED: The start of roman-numbered plain pages.
\frontmatter

% REQUIRED: Create the dissertation title page.
\makedisstitle

% REQUIRED: Create the signature page.  Add one line for each member
% of your dissertation committee, except for your advisor who is
% automatically added before the rest.
\begin{disssignatures}
\committeemember Second Member, Dept.~of Mathematics
\committeemember Third Member, Dept.~of Mathematics, Outside University
\end{disssignatures}

\disscopyright % optional

\begin{dissdedication}
  A dedication is optional.
\end{dissdedication}

\begin{dissacknowledgments} % recommended
  I wish to thank my advisor for her help and support.

  I am grateful to the members of my dissertation defense committee.
  
  I owe thanks to the faculty, to my fellow students, and to the kind
  and supportive staff of the Brandeis Mathematics Department.
\end{dissacknowledgments}

% REQUIRED: The dissertation abstract.
\begin{dissabstract}
  The GSAS limits you to up to 350 words of abstract text.
\end{dissabstract}

\begin{disspreface}
  A preface is optional.
\end{disspreface}

\tableofcontents % REQUIRED

% The \phantomsection command must be placed before any section whose
% hyperlinks don't refer to the right place.
\phantomsection
\listoffigures % optional

% The \phantomsection command must be placed before any section whose
% hyperlinks don't refer to the right place.
\phantomsection
\listoftables % optional


% REQUIRED: The start of arabic-numbered fancy pages.
\mainmatter

\chapter{Introduction}

Introductory matters for \thediss{author}'s dissertation.

% Force a new page.
\clearpage

% The label allows us to refer (\ref) to this section's number
% elsewhere.  I follow a prefix convention for labels to distinguish
% those for chapters (chap:), sections (sec:), subsections (sub:),
% figures (fig:), tables (tab:), items (itm:), and so on.
\section{First}\label{sec:first}

% Citations are used for bibliography references.
Page~\thepage\ cites Milnor's paper~\cite{MR29:634}.

\clearpage

\section{Second}

Unlike Section~\ref{sec:first}, page~\thepage\ cites de~Rham's
paper~\cite{MR16:957b}.  It also adds Figure~\ref{fig:onefoil}
\begin{figure}[htbp]
  \centering
  \includegraphics[height=2in]{onefoil}
  \caption{A Onefoil Knot}
  \label{fig:onefoil}
\end{figure}
which shows a onefoil knot in a thickened Klein bottle.  If
``pdflatex'' is used, the command ``$\backslash$includegraphics''
command inserts the ``onefoil.pdf'' file.  If ``latex'' is used, the
command inserts the ``onefoil.eps'' file.

\chapter{The Lefschetz Hyperplane Theorem}

As stated by Milnor in~\cite{MR29:634}:
\begin{theorem}[Lefschetz]\label{thm:lef1}
  Let $V$ be an algebraic variety of complex dimension $k$ which lies
  in the complex projective plane $\pCP^n$.  Let $P$ be a hyperplane
  in $\pCP^n$ which contains the singular points (if any) of $V$.
  Then the inclusion map
  \begin{align*}
    V\cap P &\into V \\
    \intertext{induces isomorphims of homology groups in dimensions
      less than $k-1$.  Furthermore, the induced homomorphism}
    \Homol_{k-1}(V\cap P;\rZ) &\onto\Homol_{k-1}(V;\rZ)
  \end{align*}
  is onto.
\end{theorem}
\begin{proof}
  The long exact sequence of the pair $(V,V\cap P)$
  \begin{align*}
    \cdots\to\Homol_{r+1}(V,V\cap P)\to\Homol_r V\cap P\to\Homol_r
    V\to\Homol_r(V,V\cap P)\to\cdots
  \end{align*}
  along with a proof that $\Homol_r(V,V\cap P)=0$ for $r\le k-1$ is
  enough.  By Lefschetz duality
  \begin{align}\label{eqn:homol1}
    \Homol_r(V,V\cap P)&\cong\Homol^{2k-r}(V-(V\cap P)).
  \end{align}
  And since $\mM=V-(V\cap P)\to\pCP^n-P\cong\rC^n$ is a non-singular
  affine algebraic variety of complex dimension $k$, it has the
  homotopy type of a $k$-dimensional CW-complex.  So the RHS of
  equation~\ref{eqn:homol1} is 0.
\end{proof}


% REQUIRED: The start of arabic-numbered plain pages.
% GSAS formatting allowance: The back may be single-spaced.
\backmatter
\singlespacing

% The \phantomsection command must be placed before any section whose
% hyperlinks don't refer to the right place.
\phantomsection
\bibliographystyle{amsplain}% Bibliography.
\bibliography{topology}

\end{document}

